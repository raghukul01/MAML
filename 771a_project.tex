
%=======================   Default Templete   ==================
\documentclass[a4paper]{article}
% \documentclass[12pt]{extreport}
\usepackage{graphicx}

% file with some default definations
\input{structure.tex}
\usepackage{listings}
\lstset{language=Python, basicstyle=\normalsize\sffamily\linespread{0.8}, numbers=left, numberstyle=\small, stepnumber=1, numbersep=5pt}
\usepackage{fancyhdr}
\usepackage{pdfpages} 
\setlength{\parindent}{0pt}

\pagestyle{fancy}
\fancyhf{}
\lhead{\textbf{\NAME}}
\chead{\textbf{Course Project Report}}
\rhead{\COURSE}

\usepackage{calrsfs}
\DeclareMathAlphabet{\pazocal}{OMS}{zplm}{m}{n}
\newcommand{\La}{\mathcal{L}}
\newcommand{\Lb}{\pazocal{L}}
\newcommand{\Tb}{\pazocal{T}}

%==================Header details======================
\newcommand\NAME{Bayesian Meta Learning}
\newcommand\ANDREWID{}
\newcommand\HWNUM{4}
\newcommand\COURSE{CS771}
%======================================================
\begin{document}

\includepdf[page={1}]{first_page.pdf}


\section*{Abstract}

Our project is on..

\subsection*{Subsection (if any) Name}

....

\section{Literature Review}

MAML has been designed to enable fast adaptation to unseen tasks by training on statistically related tasks. The algorithm is Model-Agnostic which means that it can be used on any model trained through gradient-descent.

\subsection{Problem Formulation}
The paper presents a generic formulation applicable to tasks like regression, classification and reinforcement learning. Let $f$ denote the common model for all tasks with $\theta$ being the central parameter to be learned. It maps input $x$ to output $a$. Each Task is denoted by $$\Tb: \{{L}({x_1}, {a_1}, ... , {x_H}, {a_H}),q({x_1}),q({x_{t+1}}|{x_t,a_t}),H\}$$

here L is the task specific loss function, $q(x_1)$ is the distribution of inputs, $q(x_{t+1} |x_t,a_t )$ is the transition probability for all states , and episode length is H. \\
H is specifically for non-iid data, for example the unfolding of reinforcement learning training sample. For the supervised iid classification and regression experiments, H=1.

\subsection{Algorithm}
$p(\Tb)$ is the distribution over tasks over which we would train and test the meta-learning model. We will follow a K-Shot learning setting. First, a batch of tasks is sampled from $p(\Tb)$ and for each sample task $\Tb_i$, \textit{K} inputs are sampled from $q_i$. These sampled inputs are used to learn task specific $\theta_i$s which were initialized to the central parameter $\theta$. Then, we sample \textit{K} more inputs to evaluate that particular task with the $\theta_i$s. The sum of the loss incurred in all the tasks is used as the meta-loss function to find the optimal $\theta$. 

\begin{algorithm}[H]
	\SetAlgoLined
	\KwIn{$p(\Tb):$ Distribution over tasks, $\alpha, \beta:$ Step size hyper parameters}
	\KwOut{Parameters for this model} 
	randomly initialize $\theta$ \\ 
	\While{not done} {
		Sample batch of tasks $\Tb_i \approx p(\Tb)$ \\ 
		\For{all $\Tb_i$} {
			Sample $K$ data points $D = \{ x^{(j)}, y^{(j)}\}$ from $\Tb_i$ \\  
			Compute $\nabla_\theta \Lb_{\Tb_i}(f_\theta)$ using \textit{K} samples \\ 
			Compute adapted parameters with gradient descent: \\ 
			$\theta^{'}_{i} = \theta - \alpha\nabla_\theta \Lb_{\Tb_i}(f_\theta)$ \\ 
			Sample data points $D^{'}_i = \{ x^{(j)}, y^{(j)}\}$ from $\Tb_i$ for the meta update \\ 
		}
		Update $\theta \leftarrow \theta - \beta \nabla_\theta \sum_{\Tb_i \approx p(\Tb)} \Lb_{\Tb_i} (f_(\theta^{'}_i))$ using each $D^{'}_i$ and $\Lb_{\Tb_i}$. 
	}	
	\Return{$\theta$}
\end{algorithm}

Formally, for each $\Tb_i$, we calculate $\theta_i'$ using the \textit{K} samples by the following gradient descent update.
$$ \theta_i' = \theta - \alpha \nabla_\theta \Lb_{\Tb_i}(f_\theta) $$
Here $\alpha$ is the step size and is a hyperparameter for the model. The meta objective is minimizing the sum of loss incurred in testing the sampled tasks against $D_i$s.
$$ \min_\theta \sum_{\Tb_i \sim p(\Tb)}\Lb_{\Tb_i}(f_{\theta_i'}) = \sum_{\Tb_i \sim p(\Tb) }\Lb_{\Tb_i}(f_{\theta - \alpha \nabla_\theta \Lb_{\Tb_i}(f_\theta)}) $$
Finally, we perform the meta optimization step using gradient descent for the central parameter $\theta$ on the loss function defined above. 
$$\theta \leftarrow \theta - \beta \nabla_\theta \sum_{\Tb_i \sim p(\Tb)} \Lb_{\Tb_i} (f_{\theta^{'}_i})$$

\subsection{Algorithm for supervised learning (Regression and classification)}
To formalize in the context of meta learning, the loss function used for regression is the mean squared error loss as shown below
$$ \Lb_{\Tb_{i}}(f_\phi) = \sum_{x^{(j)}, y^{(j) }\sim\Tb_{i}} || f_{\phi}(x^{(j)}) - y^{(j)} ||^2_2 $$
where $\phi$ is the model parameter.
For classification, the loss function is given as
$$ \Lb_{\Tb_{i}}(f_\phi) = \sum_{x^{(j)}, y^{(j) }\sim\Tb_{i}} y^{(j)} \log f_{\phi}(x^{(j)}) + (1-y^{(j)}) \log (1 - f_{\phi}(x^{(j)})) $$


\subsection*{Algorithm Review}

....

\subsection*{Bayesian ML}

....

\subsection*{Reinforcement Learning}

....

\pagebreak
\section{Experiments}
Most of the experiments carried out required automated differentiation through gradient update, 
for which we used TensorFlow. 
A signification computation was required in finding the second derivative, 
in the meta gradient update step, 
which involved backpropogating the meta-gradient through gradient calculation in meta gradient objective ($\theta \leftarrow \theta - \beta \nabla_\theta \sum_{T_i \approx p(T)} L_{T_i}(f_{\theta^{'}_i}) $). \\ \\
We have tested the MAML algorithm for Regression and Classfication problem. Now let's discuss each of these experiment in details.

\subsection{Regression}
We tested our model on the task of regressing from the input to output of a sine wave,
where each task had a different amplitude and phase, where amplitude varies within $[0.1, 5.0]$ and phase varies within $[0, \pi]$. 
Note that input and output in our problem are just $1$ dimensional. 
Datapoints are sampled uniformly from $[-5.0, 5.0]$ and the loss function is simply the mean sqaured error between prediction and the ground truth.

\subsubsection{Architecture}
Our regressor is a neural network model which has $2$ hidden layers of $40$ hidden units, and each each containing activation function as ReLU. \\
While training hyperparameter $\alpha = 0.01$, and our meta optimizer is Adam \cite{adam}. 

\pagebreak
....

\subsection*{Classification}

....

\section*{Conclusion}

....

\subsection*{Subsection (if any) Name}

....

\section*{Future Work}

....

\subsection*{Subsection (if any) Name}

....

\subsubsection*{Acknowledgments}

....
\pagebreak
\begin{thebibliography}{9}
\bibitem{maml}
Chelsea Finn, Pieter Abbeel and Sergey Levine.
\textit{Model-Agnostic Meta-Learning for Fast Adaptation of Deep Networks.}
2017.

\bibitem{bayesian}
Taesup Kim, Jaesik Yoon, Ousmane Dia1, Sungwoong Kim, Yoshua Bengio and Sungjin Ahn.
\textit{Bayesian Model-Agnostic Meta-Learning.}
Jun 2018.

\bibitem{gradient_based}
Erin Grant, Chelsea Finn, Sergey Levine, Trevor Darrell, Thomas Griffiths.
\textit{Recasting Gradient-based Meta-Learning as hierarchical Bayes.}
Jan 2018.

\bibitem{adam}
Maclaurin, Dougal, Duvenaud, David, and Adams, Ryan.
\textit{Gradient-based hyperparameter optimization through reversible learning.}
In International Conference on Machine Learning (ICML), 2015.
\end{thebibliography}

\end{document}
